%\htmlhr
\chapter{Index Checker\label{index-checker}}

%Runtime exceptions can cause a lot of problems in code. They can crash entire systems
%and lose important data. One of these exceptions is an IndexOutOfBoundsException.
Accessing an array or List may cause and IndexOutOfBoundsException to be thrown, as the index
being passed in may not be valid for the given array/List.
The Index Checker shows you, at compile time, what lines of code may throw an these exceptions.
%You run the Index Checker as a plugin to javac, and javac issues warnings
%about code that could throw an IndexOutOfBoundsException. This allows developers to
%detect and fix bugs early, which is cheaper and prevents run-time failures.

\section{Installation\label{index-installation}}

As we are not completely integrated into the checker framework, special steps must be taken to
install and use our checker. To install:

\begin{enumerate}
\item Download the distribution (one of the ones under downloads) at
https://github.com/gyhughes/checker-framework/releases
\item Extract the files
\item Configure your IDE, build system, or command shell to include the
Checker Framework on the classpath via instructions below:
\begin{enumerate}
\item For javac:
\begin{enumerate}
\item If you are using the bash shell, one possible setup is to add the following
to your $\sim$/.profile (or alternatively $\sim$/.bash\_profile or $\sim$/.bashrc) file:
(or just run in terminal for temporary use)

\begin{Verbatim}
// Note: you must put the path to the index checker in for PATH TO
export CHECKERFRAMEWORK=// PATH TO\ /index-checker-\IndexReleaseVersion\ HERE //
export PATH=\${CHECKERFRAMEWORK}/checker/bin:\${PATH}
\end{Verbatim}

\end{enumerate}
\item For other compilers/IDEs, check the installation section (Section~\ref{installation}) keeping in mind the
Index Checker's seperate distribution.
\end{enumerate}
\end{enumerate}

We highly recommend installing for javac, as it is the only installation we
have tested our release on. We also lightly recommend taking the first option
for setting up javac (moving to front of path), as the rest of this manual
is written assuming that this option was taken. The commands can be altered
easily if you prefer a different setup for javac.

The rest of this manual assumes that you will be using javac and that the Index
Checker's javac has been moved to the front of the path. If you have decided to
install the tool in a different manner, please do not copy the commands listed
below verbatim.

\section{Running the Checker\label{index-running}}

To run the IndexChecker, run the command

\begin{Verbatim}
javac -processor IndexChecker <JavaFileName>.java
\end{Verbatim}

For troubleshooting, see the \href{http://types.cs.washington.edu/checker-framework/current/checker-framework-manual.html#troubleshooting}
{Checker Framework manual's trouble shooting section}.

\section{Index Annotations\label{index-annotations}}

\begin{description}
\item[\refqualclass{checker/index/qual}{IndexFor(String name)}]
	indicates an int type that is within the bounds of the array with the given
	name (i >= 0 \&\& i < arr.length).
\item[\refqualclass{checker/index/qual}{IndexOrHigh(String name)}]
	indicates an int type that is possibly one over the bounds of the array with
	the given name (i >= 0 \&\& i < arr.length + 1).
\item[\refqualclass{checker/index/qual}{IndexOrLow(String name)}]
	indicates an int type that is possibly one under bounds of the array with the
	given name (i >= -1 \&\& i < arr.length).
\item[\refqualclass{checker/index/qual}{LTLength(String name)}]
	indicates an int type that is below the max bound of the array with the given
	name (i >= 0).
\item[\refqualclass{checker/index/qual}{NonNegative}]
	indicates an int type that is at least zero, which corresponds to being over
	the min bound of any array (i < arr.length).
\item[\refqualclass{checker/index/qual}{MinLen(int len)}]
	indicates an array type that is at least of length len.
\end{description}

\begin{figure}
%TODO update name
\includeimage{index-type-figure}{7cm}
\caption{The range of values represented by each type.}
\label{fig-index-figure}
\end{figure}

\begin{figure}
\includeimage{index-type-heirarchy}{7cm}
\caption{The type hierarchy of the Index Checker(subtype/supertype relationships)
The types that are grayed out should not be written by the user.}
\label{fig-index-heirarchy}
\end{figure}