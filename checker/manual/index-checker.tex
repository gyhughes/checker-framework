%\htmlhr
\chapter{Index Checker\label{index-checker}}

%Runtime exceptions can cause a lot of problems in code. They can crash entire systems
%and lose important data. One of these exceptions is an IndexOutOfBoundsException.
Accessing an array or List may cause and IndexOutOfBoundsException to be thrown, as the index
being passed in may not be valid for the given array/List. Strings are also checked, as they
are represented through an array of characters.
The Index Checker shows you, at compile time, what lines of code may throw an these exceptions.
%You run the Index Checker as a plugin to javac, and javac issues warnings
%about code that could throw an IndexOutOfBoundsException. This allows developers to
%detect and fix bugs early, which is cheaper and prevents run-time failures.

\section{Installation\label{index-install}}

As we are not completely integrated into the checker framework, special steps must be taken to
install and use our checker. To install:

\begin{enumerate}
\item Download the distribution (one of the ones under downloads) at
https://github.com/gyhughes/checker-framework/releases
\item Extract the files
\item Configure your IDE, build system, or command shell to include the
Checker Framework on the classpath via instructions below:
\begin{enumerate}
\item For javac:
\begin{enumerate}
\item If you are using the bash shell, one possible setup is to add the following
to your $\sim$/.profile (or alternatively $\sim$/.bash\_profile or $\sim$/.bashrc) file:
(or just run in terminal for temporary use)

\begin{Verbatim}
// Note: you must put the path to the index checker in for PATH TO
export CHECKERFRAMEWORK=// PATH TO\ /index-checker-\IndexReleaseVersion\ HERE //
export PATH=\${CHECKERFRAMEWORK}/checker/bin:\${PATH}
\end{Verbatim}

\end{enumerate}
\item For other compilers/IDEs, check the installation section
(Section~\ref{installation}) keeping in mind the
Index Checker's seperate distribution.
\end{enumerate}
\end{enumerate}

We highly recommend installing for javac, as it is the only installation we
have tested our release on. We also lightly recommend taking the first option
for setting up javac (moving to front of path), as the rest of this manual
is written assuming that this option was taken. The commands can be altered
easily if you prefer a different setup for javac.

The rest of this manual assumes that you will be using javac and that the Index
Checker's javac has been moved to the front of the path. If you have decided to
install the tool in a different manner, please do not copy the commands listed
below verbatim.

\section{Running the Checker\label{index-running}}

To run the IndexChecker, run the command

\begin{Verbatim}
javac -processor IndexChecker <JavaFileName>.java
\end{Verbatim}

For troubleshooting, see the \href
{http://types.cs.washington.edu/checker-framework/current/checker-framework-manual.html#troubleshooting}
{Checker Framework manual's trouble shooting section}.

\section{Index Annotations\label{index-annotations}}

\begin{description}
\item[\refqualclass{checker/index/qual}{IndexFor(String name)}]
	indicates an int type that is within the bounds of the array with the given
	name (i >= 0 \&\& i < arr.length).
\item[\refqualclass{checker/index/qual}{IndexOrHigh(String name)}]
	indicates an int type that is possibly one over the bounds of the array with
	the given name (i >= 0 \&\& i < arr.length + 1).
\item[\refqualclass{checker/index/qual}{IndexOrLow(String name)}]
	indicates an int type that is possibly one under bounds of the array with the
	given name (i >= -1 \&\& i < arr.length).
\item[\refqualclass{checker/index/qual}{LTLength(String name)}]
	indicates an int type that is below the max bound of the array with the given
	name (i >= 0).
\item[\refqualclass{checker/index/qual}{NonNegative}]
	indicates an int type that is at least zero, which corresponds to being over
	the min bound of any array (i < arr.length).
\item[\refqualclass{checker/index/qual}{MinLen(int len)}]
	indicates an array type that is at least of length len.
\end{description}

\begin{figure}
%TODO update name
\includeimage{index-type-figure}{7cm}
\caption{The range of values represented by each index type annotation.}
\label{fig-index-figure}
\end{figure}

\begin{figure}
\includeimage{index-type-heirarchy}{7cm}
\caption{The type hierarchy of the Index Checker(subtype/supertype relationships)
The types that are grayed out should not be written by the user.}
\label{fig-index-heirarchy}
\end{figure}

\section{What the Index Checker checks\label{index-checks}}
In order to get code to type check using the Index Checker, indices must have been
compared against the low and high bounds of an array before accessing it. In terms
of the annotation types, an index must gain the annotated type @IndexFor("arr") to access some
arr/list. The Index Checker will issue a warning if any of the following rules are violated:
\begin{itemize}
\item Only an @IndexFor("arr") int i can be used access arr[i]
\item Only an @IndexFor("lst") int i can be used to call lst.get(i) on some List lst
\item Only an @IndexFor("msg") int i can be used to call msg.charAt(i) on some String msg
\end{itemize}

This is accomplished through comparing an index the bounds of an array through if block
comparisons. These are primarily against <array>.length, <array>.length-1, and int literals
(or variables initiallized with them). See figure \ref{fig-index-hierarchy} for an example

\textbf{NOTE}: Currently, we only refine the type of the left hand side of the comparison.

\begin{figure}
\begin{Verbatim}
int i = 0;                // @NonNegative
int j = input.nextInt();  // @UnknownIndex, anything may be returned
if (j >= 0 && j < arr.length) {
                          // j is now @IndexFor("arr")
                          // i is still @NonNegative
    int after_j = j++;    // @IndexOrHigh("arr")

    arr[j];               // legal
    arr[after_j];         // illegal, may be one too high
    arr[i];               // illegal, never checked against arr.length
    // this is a legitimate error; arr could be an empty array
}
\end{Verbatim}
\caption{An example piece of code that demonstrates how the IndexChecker uses these annotations. \textbf{NOTE}: Currently, we only refine the type of the left hand side of the comparison.}
\label{fig-index-hierarchy}
\end{figure}

\section{Side Effecting Lists\label{index-side-effects}}
As indicies are tied to specific arrays/Lists, side-effecting a list will change type annotations
in order to maintain as much information as possible. In particular:
\begin{itemize}
\item When a variable gets re-assigned, all annotated type with that variable's name as its value
get changed to either @Unknown or @NonNegative depending on what we already know about the type.
\item When elements are added to a list, nothing happens (no invariants about being in bounds are
broken when the list grows).
\item When a single element is removed from a list, we perform a transfer rule on all types as if
we were adding 1 to that index's value.
\item When a list is cleared, we clear all annotations to either @Unknown or @NonNegative (This is
because lists can be aliased).
\item We currently do not support Iterator usage. The use of Iterators cause this checker to become
unsound and generate false negatives (missed errors).
\end{itemize}

\section{Multiple arrays/Lists\label{index-multiples}}
\begin{itemize}
\item @SomeType("a") int i > (or >=) @SomeType("b") int j:\\
i gains the type @RefinedType("a"), where the @RefinedType is the resulting type as computed by our
normal dataflow rules. The reason for this is that greater than comparisons are only good for
asserting if you are within the lower bounds of an array/list. Therefore we want to keep whatever
information about length comparisons already made.
\item @SomeType("a") int i < (or <=) @SomeType("b") int j:\\
i gains the type @RefinedType("b"). The
reason for this is that less than comparisons are only good for asserting that we are within the
higher bounds of an array/list. Therefore we want to inherit whatever new information we get out of
this comparison.
\item @SomeType("a") int i == @SomeType("b") int j:\\
i gains either @RefinedType("a") or @RefinedType("b") depending on the exact comparison being made.
Generally, the Index Checker will choose the annotation to which it knows most about. This may generate
false warnings.
\end{itemize}

\section{Bug Reports\label{index-bug-reports}}
As the Index Checker still lives on a seperate fork the the Checker Framework,
please log bugs on our \href{https://github.com/gyhughes/checker-framework/issues}
{repository's issue tracker}.
More info can be found on the \href
{https://guides.github.com/features/issues/}{github features page}.
In bug reports, please include information detailed by reporting bugs section
(Section~\ref{reporting-bugs}).