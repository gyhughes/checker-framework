%\htmlhr
\chapter{Index Checker for array and list bounds checking\label{index-checker}}

Accessing an array, list, or string causes an IndexOutOfBoundsException
if the index is out of bounds (negative or too large).
The Index Checker warns you, at compile time, about possible
IndexOutOfBoundsExceptions.


\section{Running the Index Checker\label{index-running}}

To run the Index Checker, run the command

\begin{Verbatim}
javac -processor IndexChecker <JavaFileName>.java
\end{Verbatim}


\section{Index annotations\label{index-annotations}}

\begin{description}
\item[\refqualclass{checker/index/qual}{IndexFor}(String name)]
	the value is within the bounds of the array with the given
	name (i $\ge$ 0 \&\& i < arr.length).
\item[\refqualclass{checker/index/qual}{IndexOrHigh}(String name)]
	the value is possibly one beyond the bounds of the array with
	the given name (i $\ge$ 0 \&\& i < arr.length + 1).
\item[\refqualclass{checker/index/qual}{IndexOrLow}(String name)]
	the value is possibly one before the bounds of the array with the
	given name (i $\ge$ -1 \&\& i < arr.length).
\item[\refqualclass{checker/index/qual}{LTLength}(String name)]
	the value is less than the upper bound of the array with the given
	name (i < arr.length).
\item[\refqualclass{checker/index/qual}{NonNegative}]
	the value is at least zero (i $\ge$ 0).
\item[\refqualclass{checker/index/qual}{MinLen}(int len)]
	indicates an array type whose length is at least len.
\item[\refqualclass{checker/index/qual}{UnknownIndex}]
        no information is known about the upper and lower bounds of the value.
        This annotation is used internally by the type system but should rarely
        be written by a programmer.
\item[\refqualclass{checker/index/qual}{IndexBottom}]
        the value is null.
        This annotation is used internally by the type system but should rarely
        be written by a programmer.
\end{description}

\begin{figure}
%TODO update name
\includeimage{index-type-figure}{7cm}
\caption{The range of values represented by each index type annotation.}
\label{fig-index-figure}
\end{figure}

\begin{figure}
\includeimage{index-type-hierarchy}{7cm}
\caption{The type hierarchy of the Index Checker (subtyping relationships).
The qualifiers on the left apply to integral types, and
\refqualclass{checker/index/qual}{MinLen} applies to array types.
The types that are grayed out should not be written by the user.}
\label{fig-index-heirarchy}
\end{figure}

\section{What the Index Checker checks\label{index-checks}}

The Index Checker issues a warning if any of the following rules are violated:
\begin{itemize}
\item Only an @IndexFor("arr") int i can be used access arr[i]
\item Only an @IndexFor("lst") int i can be used to call lst.get(i) on some List lst
\item Only an @IndexFor("msg") int i can be used to call msg.charAt(i) on some String msg
\end{itemize}

\begin{figure}
\begin{Verbatim}
int i = 0;                // @NonNegative
int j = input.nextInt();  // @UnknownIndex, anything may be returned
if (j >= 0 && j < arr.length) {
                          // j is now @IndexFor("arr")
                          // i is still @NonNegative
    int after_j = j++;    // @IndexOrHigh("arr")

    arr[j];               // legal
    arr[after_j];         // illegal, may be one too high
    arr[i];               // illegal, never checked against arr.length
    // this is a legitimate error; arr could be an empty array
}
\end{Verbatim}
\caption{An example piece of code that demonstrates how the IndexChecker uses these annotations. \textbf{NOTE}: Currently, we only refine the type of the left hand side of the comparison.}
\label{fig-index-hierarchy}
\end{figure}


\section{Side effects\label{index-side-effects}}

If a list is mutated (side-effected), then the Index Checker revises its
estimates of which indices are in bounds for which arrays. In particular:
\begin{itemize}
\item When a variable gets re-assigned, all annotated type with that variable's name as its value
get changed to either @Unknown or @NonNegative depending on what we already know about the type.
\item When elements are added to a list, nothing happens (no invariants about being in bounds are
broken when the list grows).
\item When a single element is removed from a list, we perform a transfer rule on all types as if
we were adding 1 to that index's value.
\item When a list is cleared, we clear all annotations to either @Unknown or @NonNegative (This is
because lists can be aliased).
\item We currently do not support Iterator usage. The use of Iterators cause this checker to become
unsound and generate false negatives (missed errors).
\end{itemize}

\section{Multiple arrays/Lists\label{index-multiples}}
\begin{itemize}
\item @SomeType("a") int i > (or >=) @SomeType("b") int j:\\
i gains the type @RefinedType("a"), where the @RefinedType is the resulting type as computed by our
normal dataflow rules. The reason for this is that greater than comparisons are only good for
asserting if you are within the lower bounds of an array/list. Therefore we want to keep whatever
information about length comparisons already made.
\item @SomeType("a") int i < (or <=) @SomeType("b") int j:\\
i gains the type @RefinedType("b"). The
reason for this is that less than comparisons are only good for asserting that we are within the
higher bounds of an array/list. Therefore we want to inherit whatever new information we get out of
this comparison.
\item @SomeType("a") int i == @SomeType("b") int j:\\
i gains either @RefinedType("a") or @RefinedType("b") depending on the exact comparison being made.
Generally, the Index Checker will choose the annotation to which it knows most about. This may generate
false warnings.
\end{itemize}
